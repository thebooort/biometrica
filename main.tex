\documentclass{beamer}
\usepackage[utf8]{inputenc}
\usepackage[spanish]{babel}
\usepackage{utopia} %font utopia imported
\usepackage{graphicx}
\usetheme{Copenhagen}
\usecolortheme{dolphin}

%------------------------------------------------------------
%This block of code defines the information to appear in the
%Title page
\title[Biométrica y privacidad] %optional
{Biométrica}

\subtitle{ Tu cuerpo amenaza tu privacidad}

\author[Bartolomé Ortiz Viso] % (optional)
{Bartolomé Ortiz Viso}

\institute[VFU] % (optional)
{
  
  Interferencias\\
 \and
  ETSIIT, Universidad de Granada
}

\date[12/05/2017] % (optional)
{Jornadas JASyP, Mayo 2017}

\logo{\includegraphics[height=1.5cm]{logo.jpg}}

%End of title page configuration block
%------------------------------------------------------------



%------------------------------------------------------------
%The next block of commands puts the table of contents at the 
%beginning of each section and highlights the current section:

\AtBeginSection[]
{
  \begin{frame}
    \frametitle{Contenido}
    \tableofcontents[currentsection]
  \end{frame}
}
%------------------------------------------------------------


\begin{document}

%The next statement creates the title page.
\frame{\titlepage}


%---------------------------------------------------------
%This block of code is for the table of contents after
%the title page
\begin{frame}
\frametitle{Contenido}
\tableofcontents
\end{frame}
%---------------------------------------------------------


\section{Sobre mí}

%---------------------------------------------------------
%Changing visivility of the text
\begin{frame}
\frametitle{¿Quién soy?}

\begin{itemize}
    \item<1-> Matemáticas (en particular matemática aplicada) \pause
    \item<2-> Informática \pause
    \item<3-> Derechos civiles / activismo \pause
    \item<4-> Curiosidad \pause
\end{itemize} 
Ya se ha acabado la parte aburrida (LOL)
\end{frame}

%---------------------------------------------------------

\section{Biométrica: Bases y conceptos}
%---------------------------------------------------------
%Example of the \pause command
\begin{frame}
\frametitle{¿Qué son los sistemas biométricos?}
\textit{"Biometrics systems are designed to identify or verify the identity of people by using their intrinsic physical or behavioral characteristics. Biometric identifiers include fingerprints; iris, face and palm prints; gait; voice; and DNA, among others."}(\citeauthor{EFF}). 
\begin{figure}
\includegraphics[scale=0.8]{primer.jpeg}
\centering
\end{figure}
\end{frame}
%---------------------------------------------------------

\begin{frame}
\frametitle{Breve nota histórica}

\begin{itemize}
    \item<1-> Era utilizada en China (mínimo desde) siglo XIV (método de indentificación entre comerciantes).
    \item<2->  Alphonse Bertillon, 1883, desarrolló el sistema antropométrico. (Identificación policial) 
    \item<3-> L.Flom, A. Safir, J. Daugman , 1936-1994, Reconocimiento del Iris
\end{itemize}
\end{frame}


%---------------------------------------------------------



\begin{frame}
\frametitle{¿Por qué usar sistemas biométricos?}
\begin{itemize}
    \item<1-> Necesidad de identificarse con precisión en internet y en la vida real 
    \item<2->  En este aspecto, el clásico Usuario/Contraseña cada vez está en mayor decadencia (muchos servicios y poca memoria) 
    \item<3-> Cumplimentar sistemas de identificación en varias etapas
\end{itemize}
\end{frame}

%---------------------------------------------------------


\begin{frame}
\frametitle{Ventajas}
    \item<1-> No puedes olvidarte de ti mismo
    \item<2-> Los datos biométricos son más difíciles de copiar que una contraseña o un PIN 
    \item<3-> Es mucho mas difícil compartir información o lograr que otros tengan acceso a ella
    \item<4-> Relativamente sencillo, cómodo y amigable
    \item<5-> Permiten establecer o trazar un seguimiento mas sencillo y con menor coste
\end{frame}
\begin{frame}

%---------------------------------------------------------

\begin{frame}
\frametitle{¿Qué sistemas se pueden ver en la actualidad?}
\begin{itemize}
    \item<1-> Huellas dactilares  
    \item<2-> Reconocimiento facial
    \item<3-> Reconocimiento de voz
    \item<4-> Escáner de Iris/Retina
\end{itemize}
\end{frame}

\begin{frame}
\frametitle{Huellas dactilares}
\begin{itemize}
 \item<1-> Opticos, ultrasónicos o capacitivos
\item<2->Usos: Servicios policiales y militares, acceso a instalaciones, acceso a teléfonos móviles, etc.
\end{itemize}
\end{frame}

\begin{frame}
\frametitle{Reconocimiento facial}
Objetivo principal: reconocimiento sin contacto.
\begin{itemize}
    \item<1-> 2D, 3D, analisis de textur facial y camaras térmicas 
    \item<2-> Objetivo principal: reconocimiento sin contacto.
    \item<3-> Usos: Vigilancia, servicios policiales y militares, farmaceuticas, etc
\end{itemize}
\end{frame}


\begin{frame}
\frametitle{Escáner de Iris y Retina}
\begin{itemize}
    \item<1-> Alto nivel de aleatoriedad en su estructura, el cual permite 266 grados de libertad que pueden ser codificados y una densidad de información de 3,4 bits por mm² de tejido
    \item<2-> Captura de imagen y procesado
    \item<3-> Ya puede usarse para vigilancia e identificación en empresas e insituciones de alta tecnología
\end{itemize}
\end{frame}


\section{La mayor amenaza para nuestra privacidad}

\frametitle{¿Ventajas?}
    \item<1-> No puedes olvidarte de ti mismo 
    \item<2-> Los datos biométricos son "más difíciles" de copiar que una contraseña o un PIN 
    \item<3-> Es mucho "mas difícil" compartir información o lograr que otros tengan acceso a ella
    \item<4-> Relativamente sencillo, cómodo y amigable
    \item<5-> Permiten establecer o trazar un seguimiento mas sencillo y con menor coste 
\end{frame}

\begin{frame}
\frametitle{¿Qué puede medir la biométrica en la actualidad?}

\end{frame}
%---------------------------------------------------------
%Highlighting text
\begin{frame}
\frametitle{Sample frame title}

In this slide, some important text will be
\alert{highlighted} beause it's important.
Please, don't abuse it.

\begin{block}{Remark}
Sample text
\end{block}

\begin{alertblock}{Important theorem}
Sample text in red box
\end{alertblock}

\begin{examples}
Sample text in green box. "Examples" is fixed as block title.
\end{examples}
\end{frame}
%---------------------------------------------------------


%---------------------------------------------------------
%Two columns
\begin{frame}
\frametitle{Two-column slide}

\begin{columns}

\column{0.5\textwidth}
This is a text in first column.
$$E=mc^2$$
\begin{itemize}
\item First item

\end{itemize}

\column{0.5\textwidth}
This text will be in the second column

\end{columns}
\end{frame}
%---------------------------------------------------------
\section{¡Manos a la obra!}



\end{document}
